% !TEX encoding =  UTF-8 Unicode

% Beginn
\documentclass[a4paper]{paper}

% Festlegungen dieser Datei
\newcommand{\Autor}{Jörg Schwartz}
\newcommand{\Titel}{Seminar Operatorentheorie: Kolmogorov Matrizen und Markov Operatoren WS 16/17}

% Eingabe, Encoding und Sprache
\usepackage[utf8]{inputenc}
\usepackage[german]{babel}
\usepackage[T1]{fontenc}


% Verschiedene Pakete und Verlinkungen
\usepackage{a4,fancyhdr,paralist,setspace,makeidx,german,lmodern,multicol}
\usepackage[colorlinks=true,
urlcolor=black,
linkcolor=black,
citecolor=black,
filecolor=black,
breaklinks=true,
pdftitle={\Titel},
pdfauthor={\Autor},
]{hyperref}
\usepackage{idxlayout}
\usepackage{mdframed}


% Mathematik-Layout
\usepackage{amsfonts,amssymb,amsmath,latexsym,amsthm,ifthen}
\usepackage{titlesec}
\usepackage[left=3.1cm,right=3.1cm,top=1.2cm,bottom=1.2cm,includeheadfoot]{geometry}

\theoremstyle{plain}
\newtheorem{satz}{Satz}[section]

% #### Gesonderte Einstellungen für Überschriften ####
\titleformat{\chapter}
 [display]
 {\centering\scshape}
 {\vspace{-40px}\large\textsc{Kapitel \thechapter}}
 {0em}
 {\LARGE}
 [\vspace{-20px}]
 \titleformat{\section}
 [hang]
 {\centering\scshape}
 {\Large\thesection}
 {0.5em}
 {\Large}
 []
 \titlespacing*{\section}{0px}{25px}{15px}

% #### Kopf- und Fußzeile ####
%\renewcommand{\chaptername}{}
\fancypagestyle{plain}{%
  \fancyhf{}
  \fancyhead[L]{
    \small
    \ifthenelse{\isodd{\value{page}}}
    {
      \ifthenelse{\isundefined{\vorl}}
      {~}
      {\fbox{Vorläufige Version!}}
    }
    {\thepage}
  }
  \fancyhead[R]{
    \small
    \ifthenelse{\isodd{\value{page}}}
    {\thepage}
    {
      \ifthenelse{\isundefined{\vorl}}
      {~}
      {\fbox{Vorläufige Version!}}
    }
  }
  \renewcommand{\headrulewidth}{0pt}
}

\pagestyle{fancy}
\fancyhf{}
\fancyhead[L]{
  \small
  \ifthenelse{\isodd{\value{page}}}
  {
    \ifthenelse{\isundefined{\vorl}}
    {~}
    {\fbox{Vorläufige Version!}}
  }
  {\thepage}
}
\fancyhead[R]{
  \small
  \ifthenelse{\isodd{\value{page}}}
  {\thepage}
  {
    \ifthenelse{\isundefined{\vorl}}
    {~}
    {\fbox{Vorläufige Version!}}
  }
}

\renewcommand{\headrulewidth}{0pt}
\renewcommand{\footrulewidth}{0pt}
%\renewcommand{\chaptermark}[1]{\markboth{#1}{}}


\newcommand{\Titelblatt}[3]{
  \begin{titlepage}
    ~\vspace{180px}
    \begin{center}
      \huge\textsc{#1}\vspace{10px}\\
      \Large{#2}\vspace{20px}\\
      \large{von \textsc{Erma Kurtagic, Jörg Schwartz}}\vspace{350px}\\
      \small Tübingen, der \today
      \normalsize
    \end{center}
  \end{titlepage}
}

% Der muss bald eliminiert werden !
\newcommand{\Aufg}{\newpage~\vspace{-5px}\begin{center}\textsc{Aufgaben zu Abschnitt \thesection}\vspace{-10px}\\~\end{center}}
\usepackage{amsfonts,amssymb,amsmath,latexsym,amsthm}
\usepackage{MnSymbol,wasysym}

\usepackage{ulem,nicefrac}
\usepackage[all]{xy}

\renewcommand{\labelenumi}{\upshape(\roman{enumi})}

%Neue Umgebungen
\theoremstyle{plain}
\newtheorem{folg}[satz]{Folgerung}
\newtheorem{lem}[satz]{Lemma}
\newtheorem{prop}[satz]{Proposition}
\newtheorem{beh}[satz]{Behauptung}
\newtheorem{axiom}[satz]{Axiom}
\newtheorem{hi}[satz]{Hilfssatz}
\newtheorem{algo}[satz]{Algorithmus}
\newtheorem{theo}{Theorem}
\newenvironment{fsatz}
  {\begin{mdframed}\begin{satz}}
  {\end{satz}\end{mdframed}}


\theoremstyle{definition}
\newtheorem{defi}[satz]{Definition}
\newtheorem{konstr}[satz]{Konstruktion}
\newtheorem{verein}[satz]{Vereinbarung}
\newtheorem{bem}[satz]{Bemerkung}
\newtheorem{bsp}[satz]{Beispiel}
\newtheorem{mem}[satz]{Erinnerung}
\newtheorem{aufg}[satz]{Aufgabe}
\newtheorem{probl}[satz]{Problemstellung}
\newtheorem{moti}[satz]{Motivation}
\newtheorem{einf}[satz]{Einführung}
\renewcommand*{\proofname}{Beweis}

%Abkürzungen
\newcommand{\sg}{\operatorname{sg}}
\newcommand{\mat}{\operatorname{Mat}}
\newcommand{\Kern}{\operatorname{Kern}}
\newcommand{\Bild}{\operatorname{Bild}}
\newcommand{\enom}{\operatorname{End}}
\newcommand{\spur}{\operatorname{Spur}}
\newcommand{\cond}{\operatorname{cond}}

\newcommand{\CF}{\operatorname{CF}}

\newcommand{\im}{\operatorname{im}}
\newcommand{\ord}{\operatorname{ord}}
\newcommand{\D}{\operatorname{D}}

\newcommand{\zop}{\operatorname{ZOp}}
\newcommand{\spop}{\operatorname{SpOp}}
\newcommand{\sprang}{\operatorname{SpRang}}
\newcommand{\zrang}{\operatorname{ZRang}}
\newcommand{\rang}{\operatorname{Rang}}
\newcommand{\Rang}{\operatorname{Rang}}
\newcommand{\rg}{\operatorname{rg}}
\newcommand{\rd}{\operatorname{rd}}
\newcommand{\eps}{\operatorname{eps}}

\newcommand{\ZOp}{\operatorname{ZOp}}
\newcommand{\SpOp}{\operatorname{SpOp}}

\newcommand{\GL}{\operatorname{GL}}
\newcommand{\diag}{\operatorname{diag}}

\newcommand{\lin}{\operatorname{Lin}}
\newcommand{\Lin}{\operatorname{Lin}}

\newcommand{\N}{\mathbb{N}}
\newcommand{\R}{\mathbb{R}}
\newcommand{\K}{\mathbb{K}}
\newcommand{\Q}{\mathbb{Q}}
\newcommand{\Z}{\mathbb{Z}}
\newcommand{\C}{\mathbb{C}}
\renewcommand{\L}{\mathbb{L}}
\newcommand{\Leb}{\mathcal{L}}
\newcommand{\Ceb}{\mathcal{C}}
\newcommand{\V}{\mathcal{V}}
\renewcommand{\P}{\mathbb{P}}
\newcommand{\I}{\mathbb{I}}

\newcommand{\E}{\mathcal{E}}
\newcommand{\F}{\mathcal{F}}
\newcommand{\B}{\mathcal{B}}
\newcommand{\A}{\mathcal{A}}
\renewcommand{\S}{\mathcal{S}}

\newcommand{\loc}{\text{loc}}
\newcommand{\sing}{\text{sing}}
\newcommand{\abs}{\text{abs}}
\newcommand{\Lip}{\on{Lip}}

\newcommand{\ol}[1]{\overline{#1}}
\newcommand{\on}[1]{\operatorname{#1}}
\newcommand{\konv}[1]{\xrightarrow[#1]{~}}
\renewcommand{\Im}{\mathrm{Im}}
\renewcommand{\Re}{\mathrm{Re}}

\newcommand{\LU}{\mathrm{LU}}
\newcommand{\RO}{\mathrm{RO}}
\renewcommand{\d}{\mathrm{d}}
\newcommand{\area}{\on{area}}
\newcommand{\vol}{\on{vol}}

% \newcommand{\bigcupdot}{\mathop{\dot{\bigcup}}}
\renewcommand{\mit}{\,\middle|\,}

\newcommand{\vertiii}[1]{{\left\vert\kern-0.25ex\left\vert\kern-0.25ex\left\vert #1\right\vert\kern-0.25ex\right\vert\kern-0.25ex\right\vert}}

\newcommand{\lb}{\vspace{6pt}\\}


\newcommand{\twomeasure}{\varbigcirc\hspace{-6pt}\textcolor{white}{\filledmedsquare}}
\newcommand{\threemeasure}{\varbigcirc}

\newcommand{\brevis}{
  \raisebox{-1px}{\ensuremath{\shortmid}}
  \hspace{-1.5mm}
  \mathlarger{
    \raisebox{-1.5px}{=}
  }
  \hspace{-5.5pt}
  \textcolor{white}{\filledmedsquare}
  \hspace{-10.5pt}
  \raisebox{-1px}{\ensuremath{\shortmid}}\hspace{-0.8mm}
}

\newcommand{\ganz}{\mathlarger{\mathlarger{\mathlarger{\fullnote}}}}
\newcommand{\halb}{\mathlarger{\mathlarger{\mathlarger{\halfnote}}}}
\newcommand{\viertel}{\mathlarger{\mathlarger{\mathlarger{\quarternote}}}}


\DeclareMathOperator*{\esssup}{ess\,sup}
\DeclareMathOperator*{\essinf}{ess\,inf}

\makeatletter
\g@addto@macro\normalsize{
  \setlength{\abovedisplayskip}{5pt}
  \setlength{\belowdisplayskip}{5pt}
}
\makeatother

\begin{document}

% Kopf- und Fußzeile
\pagestyle{fancy}
\fancyhf{}
\fancyhead[L]{\footnotesize\textsc{Seminar der Operatorentheorie, Prof. R. Nagel}}
\fancyhead[R]{\footnotesize\textsc{Universität Tübingen, WS 16/17}}
\fancyfoot[L]{\footnotesize Jörg Schwartz}
\fancyfoot[R]{}
\renewcommand{\headrulewidth}{0.5pt}
\renewcommand{\footrulewidth}{0.5pt}
\setcounter{section}{1}
\vspace{-50px}

\begin{center}
\large\textsc{Von Kolmogorov Matrizen zu Markov Operatoren}
\end{center}
\textbf{DNS} ist ein Molekül, welches das \textbf{genetische Material} in Organismen mittels einer Folge von \textbf{Nukleotide} kodiert. Als \textbf{Gen} bezeichnet man den Abschnitt einer DNS, welcher die Information zur Konstruktion eines bestimmten \textbf{Proteins} enthält. Die physische Position eines Gens  wird \textbf{Locus} genannt, die Variante eines Gens an einem Locus heißt  \textbf{Allele}.\par Die verschiedenen Allele eines spezifischen Locus lassen sich durch Nummerierung mit einer Teilmenge $\mathbb I\subseteq \N$ identifizieren. Die \textbf{Wahrscheinlichkeit} $p_i(t)$, mit der ein  Locus zu einem Zeitpunkt $t\geq 0$ die Allele $i\in\I$ trägt, lässt sich mithilfe von   \textbf{Zufallsvariablen} $X_t, t\geq0$ vermöge $p_i(t)=\P(X_t=i)$ modellieren. Dabei gilt $\sum_{i\in\I}p_i(t)=1$, d.h. $(p_i(t))_{i\in\I}$ ist eine \textbf{Verteilung}.

\par Die \textbf{Übergangswahrscheinlichkeit} $p_{i,j}(t,s)$ zweier Allele $i,j\in\I$ ist gegeben  durch die \textbf{bedingten Wahrscheinlichkeit} $p_{i,j}(t,s)=\P(X_t=j|X_s=i)$. Damit ist $(p_{ij}(t,s))_{i,j\in\I}$ \textbf{stochastische Matrix},  d.h. jede Zeile ist eine Verteilung. Der  $j$-te Eintrag einer Verteilung $y(t)=(p_i(t))_{i\in\I}$ bedingt auf die Verteilung $p_i(s)$ mit $t\geq s$ ist gegeben durch $p_j(t)=\P(X_t=j)=\sum_{i\in\I}p_i(s) p_{ij}(t,s)$. Damit definiert $P\colon \ell^1\to\ell^1, x\mapsto x\cdot(p_{ij}(t,s))_{i,j\in\I}=:Px$ eine linearen Operator mit $\|P\|=1$.
\par Die Familie $(X_t)_{t\geq0}$, welche die Verteilung der Allele eines Locus modelliert, ist Beispiel einer \textbf{zeitlich homogenen Markovkette} (ZHMK) - ein Prozess \textbf{ohne Erinnerung} - mit $p_{ij}(t+h,s+h)=p_{ij}(t,s)$ für alle $h>0$. Es genügt also, lediglich $p_{ij}(t):=p_{ij}(t,0)$ zu betrachten.


\begin{defi}
\begin{compactenum}
\item  Ein linearer Operator $P$ auf $\ell^1$ heißt   \textit{sub-Markov'sch}, falls $P(D)\subseteq D$ mit $D=\{x=(\xi_i)_{i\in\I}\in \ell^1; \xi_i\geq0, \|x\|\leq1\}$. Ist $\|x\|=1$ für alle $x\in D$, so heißt $P$ \textit{Markov'sch}.
\item Eine Matrix $Q=(q_{ij})_{i,j\in\I}$ heißt \textit{Kolmogorov Matrix}, wenn für alle Einträge $q_{ij}$ gilt, dass  $0\leq -q_{ii}< \infty$ und $q_{ij}\geq0$ für $i\neq j$ mit  $\sum_{j\in\I}q_{ij}=0$ für jede Zeile $i\in\I$.
\end{compactenum}
\end{defi}

\begin{prop} Sei  $(X_t)_{t\geq0}$  ZHMK mit Zustandsraum $\I\subseteq \N$ und Übergangswahrscheinlichkeiten $p_{i,j}(t)$. Dann ist $P(t)=(p_{ij}(t))_{i,j\in\I}$ für alle $t\geq0$  ein Markov Operator. Gilt $\lim_{t\downarrow 0}p_{i,i}(t)=1$ für alle  $i\in\I$, dann ist $(P(t))_{t\geq0}$  stark stetige Halbgruppe von Markov Operatoren. Dabei gilt:
\begin{compactenum}
   \item Für $\I$ endlich ist  $Q:=\lim_{h\downarrow 0}\frac{1}{h}\big(P(h)-I\big)$ eine Kolmogorov Matrix.
   \item Für $\I$ nicht endlich ist  $Qx:=\lim_{h\downarrow0}\frac{1}{h}\big(P(h)x-x\big)$ mit $x\in D(Q)$  \textit{Generator}.
   \end{compactenum}
\end{prop}

\begin{fsatz}[Hille-Yosida] Ein abgeschlossener, dicht definierter Operator $(Q, D(Q))$ ist Generator einer stark stetigen Halbgruppe von (sub-)Markov Operatoren genau dann, wenn:
  \begin{compactenum}
      \item Für alle $x\in \ell^1$ und $\lambda>0$ gibt es genau eine Lösung $y\in D(Q)$ mit $\lambda y- Qy = x$. 
      \item Für alle $\lambda>0$ ist die Resolventenabbildung  $\lambda (\lambda-Q)^{-1}$ (sub-)Markov Operator. 
  \end{compactenum}
  \end{fsatz}
  
\begin{moti}Sei $Q$ gegeben durch $q_{ii}=-i^2$ und $q_{i,i+1}=i^2$ mit $q_{ij}=0$ sonst. Dann beschreibt $Q$ einen \textbf{Geburtsprozess mit Intensität} $(i^2, i\geq0)$, d.h. für alle $i\geq0$ ist die  \textbf{Haltezeit} $S_i$ des zugehörigen Prozesses $(X_t)_{t\geq0}$ exponential verteilt mit Parameter $i^2$, d.h. $\P(S_i>t)=\text e^{-i^2 t}$: Der Prozess startet etwa im Zustand $i=1$ mit Haltezeit $S_1$, wobei $\P(S_1>t)=\text e^{-t}$ und springt anschließend zum Zustand $i=2$. Hier wartet er für eine Haltezeit $S_2$ mit $\P(S_2>t)=\text e^{-4t}$ und springt zum Zustand $i=3$, usw. Dabei gilt für den Erwartungswert $\mathbb E\big[\sum_{i=1}^\infty S_i\big]\leq\sum_{i=1}^\infty \mathbb E[S_i]=\sum_{i=1}^\infty\frac{1}{i^2}< \infty$. $Q$ liefert also keine Information über des Prozess nach der \textbf{Explosionszeit} $T=\sum_{i=1}^\infty S_i<\infty$. Also lassen sich mit einer solchen \textbf{explosiven} Matrix mehrere Prozesse bzw. Halbgruppen assoziieren.
\end{moti}

  
\begin{prop} Sei $(P(t))_{t\geq0}$ Halbgruppe einer ZHMK und $(Q,D(Q))$ Generator. Dann gilt:
\begin{compactenum}
\item  Für alle $i,j\in\I$ existieren $q_{ii}:=\lim_{t\downarrow 0}\frac{1}{t}[p_{ii}(t)-1]$ und $q_{ij}:=\lim_{t\downarrow 0}\frac{1}{t}p_{ij}(t)$.
\item  Ist $-q_{ii}< \infty$ und  $\sum_{j\neq i}q_{ij}=-q_{ii}$, dann gilt $e_i\in D(Q)$ mit $Qe_i=(q_{ij})_{j\in\I}$ für alle $i\in\I$.
\end{compactenum}\end{prop}
  

\begin{fsatz}Sei $Q$ Kolmogorov Matrix und $D(Q_0)=\textnormal{Lin}(e_i,i\in\I)$ mit $Q_0 e_i:=(q_{ij})_{j\in\I}$ Generator. Dann gibt es stets eine Fortsetzung von $(Q_0, D(Q_0))$, welche eine Halbgruppe von sub-Markov Operatoren erzeugt, die minimal unter allen Halbgruppen von Fortsetzungen ist.
\end{fsatz}

\begin{fsatz}
  Sei $Q$ Kolmogorov Matrix. Dann ist die von $(Q_m,D(Q_m))$ erzeugte minimale Halbgruppe $(P_m(t))_{t\geq0}$ genau dann ein Markov Operator, wenn $Q$ nicht explosiv ist.
\end{fsatz}

\end{document}